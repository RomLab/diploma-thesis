\section{MQTT protokol}
\label{app:mqtt-protokol}
Přenášené zprávy jsou tříděny do témat (tzv. topic). Každá zpráva patří právě do jednoho tématu, přičemž témata definuje přímo poskytovatel zprávy. Odběratel pak musí předem znát jméno (označení) tématu, aby se mohlo přihlásit u~centrálního serveru k jeho odběru. Odběratel nemusí znát umístění ani komunikační adresu poskytovatele zprávy. Musí jen znát komunikační adresu (umístění) centrální serveru. Témata jsou hierarchická a oddělená lomítky. Příklad struktury tématu:~„dum/patro/loznice/senzor/teplota“, lze tak přehledně roztřídit jednotlivá umístění zařízení a případné rozšiřování systému je pak snadné. Příklad schématu komunikace a struktury témat je zobrazen na obrázku \ref{fig:mqtt-protokol}. \cite{vojacek-mqtt}
\setnowidow[2]

\begin{figure}[H]
    \centering
    \def\svgwidth{\columnwidth}
    \input{images/svg/mqtt-protokol.pdf_tex}
    \caption[Základní funkční schéma MQTT komunikace.]{Základní funkční schéma MQTT komunikace. Příklad přenosu hodnot do koncových zařízení. Znak \# nahrazuje jednu či více úrovní, budou přijímány odběrateli všechny zprávy tykající se prvního patra domu.}
    \label{fig:mqtt-protokol}
\end{figure}

Obsahem zprávy není přesně definován. Nejčastěji se používá formát (způsob zápisu) dat \acrshort{json} (\textit{\acrlong{json}}), \acrshort{bjson} (\textit{\acrlong{bjson}}) nebo textové zprávy. Velikost zprávy je pak v aktuální verzi protokolu omezena na necelých 256 MB, ale vzhledem k~využití „Internet of~Things“ bývá většina zpráv mnohem menší. \cite{vojacek-mqtt}

Protokol MQTT popisuje jen samotný popis struktury přenášených zpráv, ale nedefinuje způsob přenosu. K tomu se využívá \acrshort{tcp/ip} (\textit{\acrlong{tcp/ip}}) protokol. Protokol definuje tři úrovně potvrzování zpráv \acrshort{qos} (\textit{\acrlong{qos}}). QoS 0 – zpráva je odeslána bez potvrzení a není zaručeno její doručení. QoS 1 – poskytovatel zprávy zprávu odešle a přes centrální server je od odběratelů posláno potvrzení, centrální server může poslat potvrzení, aniž by měl potvrzení od všech odběratelů (závisí na implementaci). QoS 2 – poskytovatel zprávu odešle, centrální server pošle poskytovateli zprávy potvrzení o přijetí, na kterou poskytovatel zprávy odpoví potvrzením. Centrální server zprávu smaže a potvrdí zprávou, čímž je komunikace mezi poskytovatelem zprávy a~centrálním serverem uzavřena. Tato komunikace probíhá i mezi centrálním serverem a~odběrateli. \cite{maly-mqtt}


V přihlašovací sekvenci se využívá identifikace klienta pomocí ID a pak volitelně i pomocí uživatelského jména a hesla. MQTT díky podpoře \acrshort{ssl} (\textit{\acrlong{ssl}})/\acrshort{tls} (\textit{\acrlong{tls}}) umožňuje přihlášení pomocí klientského SSL certifikátu. \cite{vojacek-mqtt}

\section{I$^2$C sběrnice}
\label{app:i2c-sbernice}
Typické zapojení sběrnice je na obrázku \ref{fig:i2c-sbernice}.

\begin{figure}[H]
    \centering
    \def\svgwidth{\columnwidth}
    \input{images/svg/i2c-sbernice.pdf_tex}
    \caption[Zapojení I$^2$C sběrnice.]{Zapojení I$^2$C sběrnice. Jedno zařízení pracuje v režimu master, ostatní zařízení v režimu slave.}
    \label{fig:i2c-sbernice}
\end{figure}

Komunikace vždy začíná START sekvencí (na SDA se vygeneruje sestupná hrana, na SCL je držena log. 1) a končí STOP sekvencí (na SDA se vygeneruje vzestupná hrana, na SCL je držena log.). SDA nesmí nikdy měnit svoji hodnotu, když je SCL v log. 1.  Přenos jednoho bitu zprávy probíhá, takže SCL je v log. 0, změní vysílač hodnotu SDA na takovou, jakou potřebuje. Poté nastaví SCL do log. 1. Se vzestupnou hranou pak přijímač čte hodnotu na SDA. Vysílač opět vrátí SCL do log. 0 a celý proces se opakuje s dalším bitem zprávy. Zpráva se skládá z 9 bitů. Prvních 8~bitů je datových a devátý bit je potvrzovací (log. 0 pro potvrzení nebo log. 1 a vysílač z toho vyrozumí, že zpráva není potvrzená). Nejjednodušší tvar zprávy se skládá ze START sekvence, 8~bitů, potvrzovací devátý bit a STOP sekvence. Prvních 7 bitů po START sekvenci tvoří adresu zařízení (každý slave má unikátní adresu, jinak dojde ke kolizi) a osmý bit rozhoduje o směru toku dat (zda se bude zapisovat log. 1 či číst log. 0), každý byte se potvrzuje devátým bitem, buď potvrzuje slave, když master posílá data nebo naopak master potvrzuje, když posílá slave. Tak to se potvrzuje až na poslední byte, tím se zařízení dozví, že komunikace končí a~má uvolnit SDA linku. Poté se odešle STOP sekvence. Zobrazení komunikace je na obrázku \ref{fig:i2c-sbernice-datova-komunikace-7bit-adresa}. \cite{dudka-i2c-relativene-jednoduse} \cite{olejar-strucny-popis-sbernice-i2c}

\begin{figure}[H]
    \centering
    \def\svgwidth{\columnwidth}
    \input{images/svg/i2c-sbernice-datova-komunikace-7bit-adresa.pdf_tex}
    \caption[Příklad I$^2$C datové komunikace se 7-bitovou adresací.]{Příklad I$^2$C datové komunikace se 7-bitovou adresací. Upraveno z~\cite{i2c-sbernice-datovy-paket-7bit-adresa}.}
    \label{fig:i2c-sbernice-datova-komunikace-7bit-adresa}
\end{figure}

Adresace je možná pomocí 7 bitů (128 unikátních adres, číslo je však poníženo ještě o speciální adresy, např. broadcast adresa apod.) nebo 10 bitů (1024 unikátních adres), zde se pak adresy přenáší ve dvou bytech (pro první byte se používá vyhrazená adresa, kde jsou uloženy dva nejvyšší bity adresy, v~druhém bytu je dolních osm bitů adresy).  \cite{dudka-i2c-relativene-jednoduse} \cite{tisnovsky-komunikace-po-seriove-sbernici-i2c}

Podle verze sběrnice je frekvenci SCL 100 kHz, 400 kHz, 1 MHz nebo až 3,4 MHz. Rychlost je pak přizpůsobena nejpomalejšímu zařízení na sběrnici. Pull-up rezistory jsou v řádech jednotek kiloohmů, s rostoucí frekvencí nebo delší vzdálenosti sběrnice se jejich velikosti volí menší. \cite{olejar-strucny-popis-sbernice-i2c}

\section{1-Wire sběrnice}
\label{app:1-wire-sbernice}
Trojvodičové zapojení sběrnice je na obrázku \ref{fig:1-wire-sbernice-tri-vodice}. Dvouvodičové zapojení je na obrázku \ref{fig:1-wire-sbernice-dva-vodice}.

\begin{figure}[H]
    \centering
    \def\svgwidth{\columnwidth}
    \input{images/svg/1-wire-sbernice-tri-vodice.pdf_tex}
    \caption{Zapojení 1-Wire sběrnice ve trojvodičovém provedení.}
    \label{fig:1-wire-sbernice-tri-vodice}
\end{figure}

\begin{figure}[H]
    \centering
    \def\svgwidth{\columnwidth}
    \input{images/svg/1-wire-sbernice-dva-vodice.pdf_tex}
    \caption{Zapojení 1-Wire sběrnice ve dvouvodičovém provedení.}
    \label{fig:1-wire-sbernice-dva-vodice}
\end{figure}

Komunikaci zahajuje vždy master reset pulsem. Dojde ke vygenerování sestupné hrany na datovém vodiči na log. 0 po dobu minimálně 480 µs. Pak master sběrnici uvolní (opět se objeví log. 1) a naslouchá. Pokud je na sběrnici připojené zařízení, tak detekuje tuto vzestupnou hranu a~po prodlevě (15–60~µs) vygeneruje na sběrnici po dobu 60–240 µs log. 0. Průběh komunikace je zobrazen na obrázku \ref{fig:1-wire-reset-vysilani-prijem-dat}a. Pokud se zařízení správně ohlásí, může master začít vysílat a přijímat data, která jsou vysílána v tzv. time slotech. Slot je dlouhý 60–120~µs a během jednoho slotu je vyslán nebo přijat jeden bit informace. Mezi jednotlivými sloty musí být minimálně 1 µs mezera, kdy je sběrnice v klidu. \cite{maly-1-wire-sbernice}

Existují 4 druhy slotů: zápis 1, zápis 0, čtení 1 a čtení 0. Sloty pro zápis slouží k tomu, aby master vyslal data do zařízení. Zápis 1 probíhá tak, že master vygeneruje na sběrnici log. 0 minimálně na 1~µs a nejpozději do 15 µs od začátku ji opět uvolní a ponechá volnou. Zápis 0 probíhá tak, že master vygeneruje na sběrnici log. 0 a ponechá ji tak po celý slot, tedy minimálně 60~µs. Zařízení vzorkuje stav na datovém vodiči zhruba 30 µs po začátek time slotu. Průběh komunikace je zobrazen na obrázku \ref{fig:1-wire-reset-vysilani-prijem-dat}b. \cite{maly-1-wire-sbernice}

Čtecí sloty inicializuje master, vygeneruje na sběrnici log. 0 na minimálně 1~µs a~opět ji uvolní. Po tomto zahájení může zařízení vyslat 1 bit, ponechá sběrnici v~klidu (log. 1) nebo je vygeneruje na log. 0. Průběh komunikace je zobrazen na obrázku \ref{fig:1-wire-reset-vysilani-prijem-dat}c. \cite{maly-1-wire-sbernice}


\begin{figure}[H]
    \centering
    \def\svgwidth{0.99\columnwidth}
    \input{images/svg/1-wire-reset-vysilani-prijem-dat.pdf_tex}
    \caption[Průběhy na sběrnici 1-Wire.]{Průběhy na sběrnici 1-Wire.
    a) Reset. b) Zápis dat. c) Čtení dat. Upraveno z \cite{1-wire-sbernice-prubehy}.}
    \label{fig:1-wire-reset-vysilani-prijem-dat}
\end{figure}

Každé zařízení má v sobě paměť \acrshort{rom} (\textit{\acrlong{rom}}), která obsahuje 64bitové unikátní číslo, které slouží k odlišení jednotlivých zařízení na sběrnici. Po RESET pulsu je třeba vyslat příkaz Match ROM, pak 64bitový kód zařízení, se kterým se má pracovat, a teprve poté se posílá příkaz. \cite{maly-1-wire-sbernice}