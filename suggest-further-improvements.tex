\chapter{Další možnosti rozšíření}
Dále jsou popsány další možnosti rozšíření systému, které nejsou součástí zadání této práce. Vzhledem k používání rekuperace v domě. Rozvést kabely s teplotními senzory do jednotlivých průduchů ve stropě. Snímat výstupní teplotu, která je ochlazena z venkovního prostředí, a tuto sníženou teplotu kompenzovat zapnutím vytápění, aby nedocházelo k poklesu teploty v místnosti, respektive její minimalizace.

V budoucnu se počítá s pořízením solárních panelů. Primárním cílem bude ohřívat otopnou vodu v centrálním zásobníku. Využít stávající systém pro přepínání kam danou energii využít, měřit získanou a spotřebovanou energii.

Doplnit záložní akumulátory v případě výpadku elektrické energie. Zejména pro čerpadla u krbů pro odvedení ohřáté vody z krbového výměníku. V~současné době krby disponují ochranou proti přehřátí spočívající v ochranném ventilu a ovládání kouřové klapky.

V rámci centrálního systému doplnit možnost kopírování teplotních plánů a usnadnit tak jejich tvorbu. 