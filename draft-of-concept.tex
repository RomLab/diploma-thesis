\chapter{Návrh konceptu řídicího systému}

\section{Popis celkového konceptu}
\label{sec:popis-celkoveho-konceptu}

Otopná soustava domu je zobrazena na obrázku \ref{fig:otopna-soustava-rez-domu}. Zdrojem tepla jsou krby ve sklepě, v přízemí a v patře s teplovodními výměníky. Krby s teplovodním výměníkem slouží k ohřevu otopné vody proudící skrz vložku krbu, které dobíjí zásobník otopné vody. V přízemí a v patře je rozdělovač podlahové topení s 12 otopnými okruhy, každý okruh se dá ovládat zvlášť. Dále je zde čerpadlo a manuální trojcestný směšovací ventil pro nastavení optimální teploty do podlahového vytápění. Druhým zdrojem tepla je plynový kondenzační kotel, který není v~současnosti pořízen, nicméně se s ním počítá do budoucna. Bude sloužit k~ohřívání otopné vody, pokud nebudou využiti krby s teplovodním výměníkem, zejména v letním období pro ohřev TUV. Oba zdroje tepla jsou pro ohřívání otopné vody do centrálního zásobníku (objem je 1 500 l). Kde je přibližně v jedné horní třetině výšky zásobníku umístěna nádoba TUV (objem je 120~l). Navržený systém řídí ovládání čerpadel u~rozdělovačů podlahové vytápění, čerpadel pro krby s~výměníkem a pohonů pro jednotlivé okruhy podlahové vytápění. K ovládání čerpadel otopných okruhů dochází při požadavku topení nebo pokud dojde k~topení v~krbech.


\begin{figure}[H]
    \centering
    \def\svgwidth{\columnwidth}
    \input{images/svg/otopna-soustava-rez-domu.pdf_tex}
    \caption{Otopná soustava v domě.}
    \label{fig:otopna-soustava-rez-domu}
\end{figure}

\newpage

\subsection{Hadwarová část}

Na obrázku \ref{fig:navrh-hardwarove-casti} je nakreslen hardwarový koncept, který je níže v textu popsán.

\begin{figure}[H]
    \centering
    \def\svgwidth{\columnwidth}
    \input{images/svg/navrh-hardwarove-casti.pdf_tex}
    \caption{ Návrh hardwarové části systému.}
    \label{fig:navrh-hardwarove-casti}
\end{figure}

Centrální jednotka je jednodeskový počítač s periferiemi jako ethernetový port, \acrshort{usb} (\textit{\acrlong{usb}}), univerzálními vstupy/výstupy, případně s alternativní funkcí pinů jako sběrnice \acrshort{i2c} (\textit{\acrlong{i2c}}) nebo dalšími typy periferií. Dále by měla disponovat dostatečnou velikostí \acrshort{ram} (\textit{\acrlong{ram}}) pamětí a relativně výkonným procesorem pro snadné zpracování vstupních/výstupních dat či povelů.

Bezdrátové nástěnné snímače prostorové teploty jsou napájeny z lokálních síťových adaptérů, každý modul má své napájení. Nástěnný snímač prostorové teploty se skládá z displeje pro zobrazení aktuální a požadované teploty a~dalších nastavení. Dále ze tří tlačítek pro vstup do menu a tlačítek pro zvýšení/snížení požadované teploty a teplotního senzoru. Komunikace s~centrální jednotkou je zajištěna pomocí WiFi modulu komunikující s WiFi routerem.

Kabelové nástěnné snímače prostorové teploty jsou napájeny pomocí switche s  \acrshort{poe} (\textit{\acrlong{poe}}). Další části jsou obdobné jako u bezdrátového řešení. Komunikace s centrální jednotkou je zajištěna pomocí zmíněného switche.

Indikátor stavů je propojen s centrální jednotkou, skládá z části indikující stavy pomocí \acrshort{led} (\textit{\acrlong{led}}) pro jednotlivé teploty naměřené v zásobníku otopné vody rozmístěné v jednotlivých částech nádrže. Dále je zde sběrnice pro komunikaci \acrshort{lcd} (\textit{\acrlong{lcd}}) displejem a~centrální jednotkou pro zobrazení teplot ze zásobníku, respektive dvou teplot ze spodní části. LED diody a LCD displej jsou umístěny u krbů.

Spínací jednotka se skládá z relé modulů pro ovládání jednotlivých čerpadel pro oběh otopné vody do otopných okruhů podlahové vytápění v jednotlivých patrech. Dále jsou zde ovládána čerpadla pro cirkulaci vody z krbových výměníků. V neposlední řadě je zde případné ovládání plynového kondenzačního kotle.

Zónový regulátor je umístěn v přízemí a v patře v rozdělovači pro jednotlivé otopné okruhy. Komunikace mezi zónovým regulátorem a centrální jednotkou je pomocí sběrnice. Zónový regulátor ovládá jednotlivé pohony, ty následně jednotlivé otopné okruhy. Pohony jsou přímo připojené na zónový regulátor.

Síťové prvky se skládají z centrálního switche, switche s POE a domácího WiFi routeru. Centrální switch sdružuje veškerou komunikaci jak z kabelových nástěnných snímačů prostorové teploty (pomocí POE switche), tak i bezdrátových (pomocí WiFi routeru). 

Teplotní senzory v zásobníku otopné vody jsou rozmístěné ve třech částech zásobníku (horní, střední a spodní část). Dále jsou teplotní senzory na kouřovodech u jednotlivých krbů pro detekci zatopení. Všechny senzory jsou napojeny na jednu sběrnici.



\subsubsection{Teplotní čidla}
Jak bylo zmíněno výše, teplotní čidla jsou potřebná na snímání teplot na kouřovodech krbů pro následné sepnutí oběhového čerpadla. Teplota na kouřovodech se může dosáhnout až 300 °C (optimální teplota se však pohybuje přibližně mezi 120 °C až 240 °C, kdy je nejvyšší účinnost krbu a hoření paliva), proto je nutné zvolit takové čidlo, které je na tyto teploty vhodné. Mezi takové teplotní čidlo patří odporový teplotní senzor (teplotní rozsahy od -240~°C až 600 °C) nebo termočlánek (teplotní rozsahy od -260 °C až 2~300~°C). %Pro zjištění teploty není nutná velmi velká přesnost, citlivost, jistým požadavkem je robustnost čidla (nejen ochrana čidla, ale i přívodních kabelů), vzhledem k umístění u krbu, kde je dosahováno vyšších teplot.

%Princip termočlánku spočívá v Seebeckově efektu, jsou-li spojeny dva vodiče z různých kovů, tak v místě spojení je generováno napětí. Velikost napětí je závislá na vnější teplotě a materiálu článku. Linearita výstupního napětí článku je závislá na typu termočlánku a rozsahu teplot.

Další teplotní senzory jsou nutné pro nástěnné teplotní snímače prostorové teploty pro každou místnost, zásobník otopné vody a venkovní čidlo. Teplotní rozsah těchto čidel nemusí být tak vysoký jako u měření teplot na kouřovodech. Teplotní rozsah stačí v řádů desítek stupňů. %Vzhledem ke vzdálenostem teplotních čidel a centrální jednotky bude lepší zvolit digitální teplotní čidla, které výslednou změřenou teplotu zpracuje pošle po sběrnici v digitální podobě. Není pak nutná další elektronika pro zpracování hodnot teploty jako například u termočlánku či teplotně odporového čidla.

\subsection{Komunikační část}

Komunikační koncept je nakreslen na obrázku \ref{fig:navrh-softwarove-casti}, který je níže v textu popsán.

\begin{figure}[H]
    \centering
    \def\svgwidth{\columnwidth}
    \input{images/svg/navrh-softwarove-casti.pdf_tex}
    \caption{Návrh komunikační části systému.}
    \label{fig:navrh-softwarove-casti}
\end{figure}

Komunikace mezi centrální řídicí jednotkou a bezdrátovými i kabelovými nástěnnými snímači prostorové teploty je zajištěna pomocí protokolu MQTT. Centrální jednotka dostává informace z jednotlivých nástěnných snímačů prostorové teploty, zároveň je možné některé parametry nastavovat přímo přes centrální jednotku, která následně dané nastavení pošle do daných zařízení.

Indikátor stavů komunikuje s centrální jednotkou pomocí sběrnice I$^2$C pro zobrazení hodnot na LCD displeji. Zároveň je zde připojení na vstupní/výstupní piny centrální jednotky pro ovládání indikačních LED diod.

Spínací jednotka je propojena s centrální jednotkou pro ovládání čerpadel podlahové vytápění, čerpadel pro krbové výměníky a~kondenzačního plynového kotle.

Zónový regulátor komunikuje s centrální jednotkou pomocí I$^2$C sběrnice, následné ovládání pohonů pro otopné okruhy je přímo zónovým regulátorem.

Teplotní senzory umístěné v zásobníku otopné vody a na kouřovodech krbů komunikují s centrální jednotkou pomocí 1-Wire sběrnice.



\subsubsection{MQTT protokol}
\label{sec:mqtt-protokol}

\acrshort{mqtt} (\textit{\acrlong{mqtt}}) je jednoduchý a nenáročný \acrshort{m2m} (\textit{\acrlong{m2m}})/„Internet of Things“ komunikační protokol. Protokol je založen na principu předávání zpráv mezi klienty přes centrální server (broker). Centrální server přijímá zprávy od poskytovatele zprávy (tzv. publisher), které následně předává k přečtení čtenářům, kteří tuto zprávu odebírají (tzv. subscribers). Poskytovatel zprávy obvykle představuje nějaký senzor či měřící jednotku, která vysílá naměřeného hodnoty na centrální server, zatímco odběratel obvykle tvoří nějaká řídící jednotka, která hodnoty odebírá (přijímá) a dále s nimi pracuje nebo je zobrazuje.

Přenášené zprávy jsou tříděny do témat (tzv. topic). Každá zpráva patří právě do jednoho tématu, přičemž témata definuje přímo poskytovatel zprávy. Odběratel pak musí předem znát jméno (označení) tématu, aby se mohlo přihlásit u~centrálního serveru k jeho odběru. Odběratel nemusí znát umístění ani komunikační adresu poskytovatele zprávy. Musí jen znát komunikační adresu (umístění) centrální serveru. Témata jsou hierarchická a oddělená lomítky. Příklad struktury tématu:~„dum/patro/loznice/sensor/teplota“, lze tak přehledně roztřídit jednotlivá umístění zařízení a případné rozšiřování systému je pak snadné. Příklad schématu komunikace a struktury témat je zobrazen na obrázku \ref{fig:mqtt-protokol}.
\setnowidow[2]

\begin{figure}[H]
    \centering
    \def\svgwidth{\columnwidth}
    \input{images/svg/mqtt-protokol.pdf_tex}
    \caption[Základní funkční schéma MQTT komunikace.]{Základní funkční schéma MQTT komunikace. Příklad přenosu hodnot do koncových zařízení. Znak \# nahrazuje jednu či více úrovní, budou přijímány odběrateli všechny zprávy tykající se prvního patra domu.}
    \label{fig:mqtt-protokol}
\end{figure}

Obsahem zprávy není přesně definován. Nejčastěji se používá formát (způsob zápisu) dat \acrshort{json} (\textit{\acrlong{json}}), \acrshort{bjson} (\textit{\acrlong{bjson}}) nebo textové zprávy. Velikost zprávy je pak v aktuální verzi protokolu omezena na necelých 256 MB, ale vzhledem k využití „Internet of~Things“ bývá většina zpráv mnohem menší.

Protokol MQTT popisuje jen samotný popis struktury přenášených zpráv, ale nedefinuje způsob přenosu. K tomu se využívá \acrshort{tcp/ip} (\textit{\acrlong{tcp/ip}}) protokol. Protokol definuje tři úrovně potvrzování zpráv \acrshort{qos} (\textit{\acrlong{qos}}). QoS 0 – zpráva je odeslána bez potvrzení a není zaručeno její doručení. QoS 1 – poskytovatel zprávy zprávu odešle a přes centrální server je od odběratelů posláno potvrzení, centrální server může poslat potvrzení, aniž by měl potvrzení od všech odběratelů (závisí na implementaci). QoS 2 – poskytovatel zprávu odešle, centrální server pošle poskytovateli zprávy potvrzení o přijetí, na kterou poskytovatel zprávy odpoví potvrzením. Centrální server zprávu smaže a potvrdí zprávou, čímž je komunikace mezi poskytovatelem zprávy a~centrálním serverem uzavřena. Tato komunikace probíhá i mezi centrálním serverem a~odběrateli.


V přihlašovací sekvenci se využívá identifikace klienta pomocí ID a pak volitelně i pomocí uživatelské jména a hesla. MQTT díky podpoře \acrshort{ssl} (\textit{\acrlong{ssl}})/\acrshort{tls} (\textit{\acrlong{tls}}) umožňuje přihlášení pomocí klientského SSL certifikátu.
\setnowidow[2]
\subsubsection{I$^2$C sběrnice}
Jedná se o sériovou, synchronní a poloduplexní sběrnici. Komunikace probíhá na dvou vodičích, jeden tvoří hodinový vodič \acrshort{scl} (\textit{\acrlong{scl}}) a~datový vodič \acrshort{sda} (\textit{\acrlong{sda}}). Vodiče jsou sdílené mezi připojenými zařízeními, proto je možné aby kdokoliv komunikoval s kýmkoliv (komunikace je v této konfiguraci náročnější na zpracování). Typické zapojení sběrnice je v konfiguraci jeden master, který veškerou komunikaci řídí, a několik zařízení slave, viz obrázek \ref{fig:i2c-sbernice}. Nicméně existuje varianta s více mastery, existuje sada pravidel, jak se musí chovat, aby mohly na sběrnici pracovat společně a neovlivňovaly se. Na vodičích SCL a SDA je připojen pull-up rezistor (R), v neutrálním stavu je na vodičích log. 1. Připojená zařízení po sběrnici komunikují pomocí otevřeného kolektoru (mohou sběrnici stáhnout k zemi (log. 0), po odpojení je na sběrnici opět log. 1).

\begin{figure}[H]
    \centering
    \def\svgwidth{\columnwidth}
    \input{images/svg/i2c-sbernice.pdf_tex}
    \caption[Zapojení I$^2$C sběrnice.]{Zapojení I$^2$C sběrnice. Jedno zařízení pracuje v režimu master, ostatní zařízení v režimu slave.}
    \label{fig:i2c-sbernice}
\end{figure}

Komunikace vždy začíná START sekvencí (na SDA se vygeneruje sestupná hrana, na SCL je držena log. 1) a končí STOP sekvencí (na SDA se vygeneruje vzestupná hrana, na SCL je držena log.). SDA nesmí nikdy měnit svojí hodnotu, když je SCL v log. 1.  Přenos jednoho bitu zprávy probíhá, takže SCL je v log. 0, změní vysílač hodnotu SDA na takovou, jakou potřebuje. Poté nastaví SCL do log. 1. Se vzestupnou hranou pak přijímač čte hodnotu na SDA. Vysílač opět vrátí SCL do log. 0 a celý proces se opakuje s dalším bitem zprávy. Zpráva se skládá z 9 bitů. Prvních 8 bitů je datových a devátý bit je potvrzovací (log. 0 pro potvrzení nebo log. 1 a vysílač z toho vyrozumí, že zpráva není potvrzená). Nejednodušší tvar zprávy se skládá ze START sekvence, 8 bitů, potvrzovací devátý bit a STOP sekvence. Prvních 7 bitů po START sekvenci tvoří adresu zařízení (každý slave má unikátní adresu, jinak dojde ke kolizi) a osmý bit rozhoduje o směru toku dat (zda se bude zapisovat log. 1 či číst log. 0), každý byte se potvrzuje devátým bitem, buď potvrzuje slave, když master posílá data nebo naopak master potvrzuje, když posílá slave. Tak to se potvrzuje až na poslední byte, tím se zařízení dozví, že komunikace končí a~má uvolnit SDA linku. Poté se odešle STOP sekvence. Zobrazení komunikace je na obrázku \ref{fig:i2c-sbernice-datova-komunikace-7bit-adresa}.

\begin{figure}[H]
    \centering
    \def\svgwidth{\columnwidth}
    \input{images/svg/i2c-sbernice-datova-komunikace-7bit-adresa.pdf_tex}
    \caption[Příklad I$^2$C datové komunikace se 7-bitovou adresací.]{Příklad I$^2$C datové komunikace se 7-bitovou adresací. Upraveno z~\cite{i2c-sbernice-datovy-paket-7bit-adresa}.}
    \label{fig:i2c-sbernice-datova-komunikace-7bit-adresa}
\end{figure}

Adresace je možná pomocí 7 bitů (128 unikátních adres, číslo je však poníženo ještě o speciální adresy, např. broadcast adresa apod.) nebo 10 bitů (1024 unikátních adres), zde se pak adresy přenáší ve dvou bytech (pro první byte se používá vyhrazená adresa, kde jsou uloženy dva nejvyšší bity adresy, v~druhém bytu je dolních osm bitů adresy).

Podle verze sběrnice je frekvenci SCL 100 kHz, 400 kHz, 1 MHz nebo až 3,4 MHz. Rychlost je pak přizpůsobena nejpomalejšímu zařízení na sběrnici. Pull-up rezistory jsou v řádech jednotek kiloohmů, s rostoucí frekvencí nebo delší vzdálenosti sběrnice se jejich velikosti volí menší.




\subsubsection{1-Wire sběrnice}
\label{sec:1-wire-sbernice}
Jedná se o sériovou, asynchronní a poloduplexní sběrnici. Komunikace probíhá na jednom vodiči, dalšími vodiči jsou napájení (V$_{DD}$) a zem (GND) to je v~případě konfigurace pomocí tří vodičů (obrázek \ref{fig:1-wire-sbernice-tri-vodice}), další typ konfigurace sběrnice je pomocí jen dvou vodičů, kde napájení a komunikace probíhá na jednom vodiči, druhý vodič je zem (obrázek \ref{fig:1-wire-sbernice-dva-vodice}), během neutrálního stavu na sběrnici (log. 1) dochází k~nabíjení interního kondenzátoru, který se následně chová jako zdroj energie při log. 0 na sběrnici (komunikace), v~tomto režimu je nutné splnit vhodné podmínky pro napájení a časování pro správnou komunikaci. Sběrnice se skládá z řídícího obvodu master a jednoho či více připojených zařízení slave. Na vodiči data je připojen pull-up rezistor (R), v~neutrálním stavu je na vodiči log. 1. Připojená zařízení po sběrnici komunikují pomocí otevřeného kolektoru (mohou sběrnici stáhnout k zemi (log. 0), po odpojení je na sběrnici opět log. 1).

Komunikaci zahajuje vždy master reset pulsem. Dojde ke vygenerování sestupné hrany na datovém vodiči na log. 0 po dobu minimálně 480 µs. Pak master sběrnici uvolní (opět se objeví log. 1) a naslouchá. Pokud je na sběrnici připojené zařízení, tak detekuje tuto vzestupnou hranu a po prodlevě (15–60~µs) vygeneruje na sběrnici po dobu 60–240 µs log. 0. Průběh komunikace je zobrazen na obrázku \ref{fig:1-wire-reset-vysilani-prijem-dat}a. Pokud se zařízení správně ohlásí, může master začít vysílat a přijímat data, která jsou vysílána v tzv. time slotech. Slot je dlouhý 60–120~µs a během jednoho slotu je vyslán nebo přijat jeden bit informace. Mezi jednotlivými sloty musí být minimálně 1 µs mezera, kdy je sběrnice v klidu. 

Existují 4 ruhy slotů: zápis 1, zápis 0, čtení 1 a čtení 0. Sloty pro zápis slouží k tomu, aby master vyslal data do zařízení. Zápis 1 probíhá tak, že master vygeneruje na sběrnici log. 0 minimálně na 1 µs a nejpozději do 15 µs od začátku ji opět uvolní a ponechá volnou. Zápis 0 probíhá tak, že master vygeneruje na sběrnici log. 0 a ponechá ji tak po celý slot, tedy minimálně 60~µs. Zařízení vzorkuje stav na datovém vodiči zhruba 30 µs po začátek time slotu. Průběh komunikace je zobrazena na obrázku \ref{fig:1-wire-reset-vysilani-prijem-dat}b.

Čtecí sloty inicializuje master, vygeneruje na sběrnici log. 0 na minimálně 1~µs a~opět ji uvolní. Po tomto zahájení může zařízení vyslat 1 bit, ponechá sběrnici v~klidu (log. 1) nebo je vygeneruje na log. 0. Průběh komunikace je zobrazena na obrázku \ref{fig:1-wire-reset-vysilani-prijem-dat}c.

Každé zařízení má v sobě paměť \acrshort{rom} (\textit{\acrlong{rom}}), která obsahuje 64bitové unikátní číslo, které slouží k odlišení jednotlivých zařízení na sběrnici. Po RESET pulsu je třeba vyslat příkaz Match ROM, pak 64bitový kód zařízení, se kterým se má pracovat, a teprve poté se posílá příkaz.


\begin{figure}[H]
    \centering
    \def\svgwidth{\columnwidth}
    \input{images/svg/1-wire-sbernice-tri-vodice.pdf_tex}
    \caption{Zapojení 1-Wire sběrnice ve trojvodičovém provedení.}
    \label{fig:1-wire-sbernice-tri-vodice}
\end{figure}

\begin{figure}[H]
    \centering
    \def\svgwidth{\columnwidth}
    \input{images/svg/1-wire-sbernice-dva-vodice.pdf_tex}
    \caption{Zapojení 1-Wire sběrnice ve dvouvodičovém provedení.}
    \label{fig:1-wire-sbernice-dva-vodice}
\end{figure}

\newpage

\begin{figure}[H]
    \centering
    \def\svgwidth{0.99\columnwidth}
    \input{images/svg/1-wire-reset-vysilani-prijem-dat.pdf_tex}
    \caption[Průběhy na sběrnici 1-Wire.]{Průběhy na sběrnici 1-Wire.
    a) Reset. b) Zápis dat. c) Čtení dat. Upraveno z \cite{1-wire-sbernice-prubehy}.}
    \label{fig:1-wire-reset-vysilani-prijem-dat}
\end{figure}

\section{Řídicí systém}
V současné době existuje poměrně dost open-source projektů pro monitorování a ovládání domácí automatizaci. Do které lze zařadit inteligentní řízení vytápění. Mezi velké projekty lze jmenovat systém Home Assistant a OpenHab. Oba jsi jsou poměrně podobní, liší programovacím jazykem, který je použit pro jejich systémové jádro, dále syntaxí pro zápis automatizací, množstvím integrovatelných zařízení (vytvořené \acrshort{api} (\textit{\acrlong{api}}) pro snadné spárování), vydáváním aktualizací, složitostí vytváření či přidávání zařízení do systému, přehlednou a dostupnou dokumentací a uživatelskou základnou, případně dalšími vlastnostmi. Na základě zkušenosti se systémem Home Assistant jak z pohledu dobré zkušenosti ze strany komunity, široké nabídky možnosti nastavení a relativně rychlou tvorbou automatizace jsem tento systém zvolil pro řízení vytápění rodinného domu.

\subsection{Home Assistant}
Home Assistant (dále jen \acrshort{ha}) je systém naprogramovaný v jazyce Python 3 a~podporuje mnoho technologií používaných v oblasti domácí automatizace. HA podporuje několik stovek zařízení či služeb (obecně komponent) od desítek velkých firem. Přesněji sdružuje jejich společné ovládání a vzájemnou propojenost automatizací. Vše je tak na jednom místě a možné ovládat přes jednoduché grafické rozhraní.

Všechna data jsou uložena na vlastním úložišti, tedy vlastním počítači, nasu, Raspberry Pi apod. Není tedy potřeba zakládat účet pro využívání služeb (některé služby však potřebují internetové připojení pro stahování informací např. předpověď počasí) a posílat data třetím stranám.

Systém se skládá ze samotné aplikace HA a z operačního systému na kterém HA běží. HA je možné nainstalovat na systém Linux, Windows, macOS. Též je přímá oficiální podpora pro Raspberry Pi, Asus Tinkerboard, Odroid a~Intel NUC, nicméně funkčnost lze najít i pro jiná zařízení. Existují čtyři varianty instalace systému, liší se nutnými zkušenostmi pro správu HA tak i~operačního systému, možnostmi správy aktualizací či obnovování, vracení nastavení, dále způsoby zálohování, možnostmi operačního systému (zda je předinstalován omezený OS nebo se jedná o plnohodnotnou verzi) v~neposlední řadě, zda je využit kontejner Docker či je HA nainstalován přímo v operačním systému, nebo lépe při využití virtuálního prostředí.

\subsubsection{Architektura Home Assistantu}
Obecně není stanoven otevřený standard pro komunikaci inteligentních zařízení. Tato skutečnost zamezuje vzájemnou komunikaci mezi zařízeními a~především většina zařízení není určena k řízení jiných zařízení. V HA se takové zařízení, která spravuje všechny ostatní nazývá \textbf{rozbočovač}.

Minimum, co by rozbočovač měl umět, je sledovat stav připojených zařízení a schopnost je řídit. Například u světel nás zajímá informace, zda jsou rozsvícená či nikoliv a umožnit změnit jejich stav. U senzoru sledujeme jeho hodnotu. Rozbočovač s~těmito možnostmi umožňuje \textbf{řízení domácnosti}.

Jistým krokem k domácí automatizaci je spuštění \textbf{uživatelsky nadefinovaných nastavení} na základě informací z domácí vrstvy řízení (například zatažení žaluzií při nadměrném osvícení slunečními paprsky). Rozbočovač s~těmito schopnostmi je schopný \textbf{domácí automatizace}.

Poslední kategorie, která je stále v budoucnu se nazývá \textbf{chytrý domov}. Samoučící a adaptivní systém, který rozhoduje, která událost by měla ovlivnit jiná zařízení.

Výše popsaný přehled řízení domácí automatizace HA je na obrázku \ref{fig:ha-prehled-domaci-autmatizace}.


\begin{figure}[H]
    \centering
    \def\svgwidth{\columnwidth}
    \input{images/svg/ha-prehled-domaci-autmatizace.pdf_tex}
    \caption[Přehled řízení domácí automatizace HA.]{Přehled řízení domácí automatizace HA. Upraveno z \cite{home-assistant-architektura}.}
    \label{fig:ha-prehled-domaci-autmatizace}
\end{figure}

\subsubsection{Jádro architektury Home Assistant}
Jádro HA odpovídá za řízení domácnosti. Skládá ze čtyř části, které to umožňují (obrázek \ref{fig:ha-jadro-architektury}):

\begin{itemize}
\item Sběrnice událostí – umožňuje vyvolání a poslech událostí – „srdce“ HA.
\item Stavový stroj – sleduje stav zařízení a spustí \textbf{změnu stavu} událostí, pokud došlo ke změně.
\item  Registr služeb – poslouchá sběrnici událostí pro \textbf{volání služby} událostí a umožňuje jinému kódu registrovat služby.
\item Časovač – posílá události \textbf{změny času} každou jednu sekundu na sběrnici událostí.
\end{itemize}

\begin{figure}[H]
    \centering
    \def\svgwidth{\columnwidth}
    \input{images/svg/ha-jadro-architektury.pdf_tex}
    \caption[Jádro architektury HA.]{Jádro architektury HA. Upraveno z \cite{home-assistant-architektura}.}
    \label{fig:ha-jadro-architektury}
\end{figure}

\subsubsection{Architektury komponent}
HA je možné rozšiřovat přes tzv. komponenty. Každá komponenta odpovídá za určitou oblast v rámci HA. Komponenty mohou poslat spouštěcí události, nabízet služby a řídit/měnit stavy. Komponenty jsou napsány v Pythonu. Sám HA nabízí několik stovek takovýchto komponent k použití. Znázornění využití komponent je na obrázku \ref{fig:ha-architektura-komponent}.

\begin{figure}[H]
    \centering
    \def\svgwidth{\columnwidth}
    \input{images/svg/ha-architektura-komponent.pdf_tex}
    \caption[Znázornění využití komponent v~HA.]{Znázornění využití komponent v HA. Upraveno z \cite{home-assistant-architektura}.}
    \label{fig:ha-architektura-komponent}
\end{figure}

Jsou zde dva typy komponent. První typ, které interagují se zařízeními „Internet of Things“ (například inteligentní žárovky). Druhý typem jsou komponenty, které reagují na událost ke kterým dojde v~HA (například nastavená automatizace).


\subsection{Inteligentní část systému}
Pro co největší využití centrálního řízení podlahového vytápění je vhodné využít různé metody pro její optimalizaci, což se následně promítne do nákladů energie, taktéž i do teplotního komfortu uživatelů. Velmi častá situace je, že domy jsou vytápěny podle momentální teploty. Toto řešení není ideální, zejména v zateplených domech, případně s podlahovým topením. Problémem jsou hlavně podzimní a zimní dny, kdy teplota nad ránem prudce klesne. Reakce vytápěcího systému je poměrně rychlá a začne přitápět. Vzhledem k~setrvačnosti otopné soustavy a to především u podlahového topení dojde k~pomalé teplotní změně, než se dané nastaví projeví, ranní mrazík mezitím zmizí. Opačný problém může nastat odpoledne, kdy začnou sluneční paprsky pražit do oken, čímž máme nepříjemně přetopeno. Výsledkem je nepříjemný uživatelský komfort a zbytečná platba za energie.

Jednou z metod je využití předpovědi počasí, kdy dopředu víme teplotní předpověď, kterou můžeme začlenit do teplotních programů (časově nastavený úsek pro vytápění) definované uživatel a na základě  předpovědi se rozhodnout, zda je nutné v místnosti přitápět dříve v případě snížení venkovní teploty nebo naopak s vytápěním počkat. 

Samoučící funkcí lze dosáhnout pro každou místnost optimální zahájení vytápění, kdy systém si danou místnost „osahá“ a rozhodne, jak dlouho bude vytápění trvat na danou teplotu. Tím lze eliminovat nepříjemný uživatelský komfort, kdy v daný čas není nastavena požadovaná teplota.

Výhodnou funkcí je detekce otevřeného okna. V případě otevření okna, dojde k poklesu vnitřní teploty místnosti, tento pokles lze vyhodnotit a~lze tak zakázat vytápění pro danou místnost, dojde  tak k úspoře zbytečně vynaložených nákladů.

Co se týče nastavení teplot pro vytápění, jsou zde dvě možnosti, využití takzvaného manuálního režimu, kdy na základě nastavené teploty se vytápění jednotlivé místnosti, uživatel si musí vytápění zapínat na základě svých potřeb (tím se značně eliminuje inteligentní část vytápění), lze daný režim rozšířit o zapínání  v daný čas a topit po definovanou dobu. Druhou možností je vytápění podle uživatelsky definovaných časových pásmech po celý týden, tím lze zajistit optimální vytápění pro konkrétní hodiny, kdy se v domě někdo nachází, vše je automatizované podle všedních zvyklostí. Dalšími možnostmi je například snížení teploty v noci na uživatelsky komfortní teplotu, kdy dochází k temperování teploty po celou noc na nižší teploty, čím lze v ranních hodinách  zajistit poměrně rychlé vytopení na danou teplotu pro ranní vstávání a zajistit, tak příjemný ranní teplotní komfort. V období, kdy dům po určitou dobu nikdo neobývá, zejména v období dovolené, lze nastavit režim dovolená a temperovat dům na nižší teploty, po návratu opět dojde k přenastavení do klasického režimu. 

Další nutná funkce pro řízení je dobíjení TUV. Tato volba se hlavně týká teplých měsíců. Proto je nutné mít podobné režimy pro dobíjení jako výše popsané pro vytápění.

Pokud je v domě více zdrojů tepla, pak je nutné přihlídnout k provozní ceně těchto zdrojů, zejména tedy použitého paliva. V mém případě se jedné o plynový kondenzační kotel (zatím ještě nepořízen) a krby s teplovodním výměníkem. Je nutné optimalizovat, kdy se jaký zdroj má použít. Primárním cílem je použití krbů, kvůli současné cenně dřeva. Proto je nutné upozorňovat uživatele, zda je nutné topit, například podle teplotních plánů či naopak přestat topit kvůli naakumulovaní celého zásobníku otopné vody. V případech, kdy uživatel nezačal topit (z důvodu, že není přítomen nebo se jedná o noc), pak systém by měl rozhodnout, zda použije plynový kondenzační kotel, který je samoobslužný.

