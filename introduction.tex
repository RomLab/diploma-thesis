\chapter{Úvod}

V současné době s rozvojem elektroniky jsou k dipozici nové možnosti domácí automatizace různého druhu. Cílem této automatizace je ekonomické, energetické řízení, víceúčelové použití a~rekonfigurace nastavení a to vše pro potřeby obyvatel s cílem zvýšit jejich pohodlí.


Jednou ze zajímavých oblastí této automatizace je vytápění domácnosti. Na dnešním trhu je možné nalézt mnoho výrobců tohoto řešení. Všichni však mají stejný primární cíl dosáhnout požadované teploty v místnosti, co možná s největšími úspory na spotřebované energii. To ve většině případech dosahují podle nastaveného teplotního režimu od uživatele. Existují i~takové, které si tento režim udělají samy podle aktivit obyvatel. Vzhledem k~budování nízkoenergetický, pasivních či nulových domů je optimální vytápění nezbytností.

Jako další alternativa pro řešení oblasti vytápění domácnosti vznikla tato práce. Podobná tématika byla i v bakalářské práci. Kde automatizace vytápění byla aplikována na starším rodinném domě s centrálním termostatem, automatickým peletovým kotlem, deskovými otopnými tělesy a~se zásobníkem teplé užitkové vody. V diplomové práci využívám stejný řídicí software pro vytápění, ale jedná se o novostavbu s podlahovým vytápěním, zónovou regulací, centrálním zásobníkem otopné vody (dále jen \acrshort{zov}) se zabudovaným zásobníkem teplé užitkové vody. Jako zdroje tepla jsou použity krby s teplovodním výměníkem a plynový kondenzační kotel.

Práce je rozdělena na několik části. V~první teoretické části jsou uvedeny informace o podlahovém vytápění a komerčních produktech. Dále hardwarový koncept celého řídicího systému a nutných zařízení pro zónovou regulaci vytápění. Komunikační část mezi centrální jednotkou a akčními členy pro řízení jednotlivých otopných okruhů. Popis řídicího systému Home Assistant včetně inteligentní části. V druhá praktické části je výběr komponent/zařízení a popis dílčích částí celého systému včetně konstrukční části. V neposlední řadě též popis softwaru, jak pro jednotlivé zařízení, tak i řídicího systému. Na konci jsou navrhnuty vylepšení a závěr.

\newpage
\section{Cíle práce}
\begin{itemize}
\item Prostudujte problematiku podlahového vytápění při využití zónové regulace a její principy.
\item Navrhněte koncept centrální jednotky a dalších nutných zařízení pro zónovou regulace vytápění.
\item Navrhněte koncept komunikace centrální jednotky, lokálních nástěnných snímačů prostorové teploty a akčních členů pro řízení jednotlivých topných okruhů.
\item Vyberte nebo zhotovte zařízení pro ovládání jednotlivých částí zónové regulace vytápění a lokální snímače prostorové teploty.
\item V řídicím systému využijte inteligentní část pro vytápění.
\item Ověřte funkčnost celého systému. Navrhněte případná vylepšení.

\end{itemize}