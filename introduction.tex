\chapter{Úvod}

V současné době s rozvojem elektroniky jsou k dipozici nové možnosti domácí automatizace různého druhu. Cílem této automatizace je ekonomické, energetické řízení, víceúčelové použití a rekonfigurace nastavení a to vše pro potřeby obyvatel s cílem zvýšit jejich pohodlí.


Jednou ze zajímavých oblastí této automatizace je vytápění domácnosti. Na dnešním trhu je možné nalézt mnoho výrobců tohoto řešení. Všichni však mají stejný primární cíl dosáhnout požadované teploty v místnosti, co možná s největšími úspory na spotřebované energii. To ve většině případech dosahují podle nastaveného teplotního režimu od uživatele. Existují i~takové, které si tento režim udělají sami podle aktivit obyvatel. Vzhledem k~budování nízkoenergetický, pasivních či nulových domů je optimální vytápění nezbytností.

Jako další alternativa pro řešení oblasti vytápění domácnosti vznikla tato práce. Podobnou tématikou jsem se zabýval i ve své bakalářské práci. Kde jsem automatizaci vytápění aplikoval na starším rodinném domě s centrálním termostatem, automatickým peletovým kotlem, deskovými radiátory a se zásobníkem teplé užitkové vody. V diplomové práci využívám stejný řídicí software pro vytápění, ale jedná se novostavbu s podlahovým vytápěním, zónovou regulací, centrálním zásobník otopné vody (se zabudovaným zásobníkem teplé užitkové vody), jako zdroje tepla jsou použity krby s teplovodním výměníkem a plynový kondenzační kotel.

Současná verze dokumentu je charakteru semestrálního projektu, která je teoretickým podkladem k diplomové práci a bude její součástí. Cíle tohoto semestrálního projektu jsou popsány níže. Práce je nyní rozdělena na tři části. V~první části jsou uvedeny informace o podlahovém vytápění a komerčních produktech. V druhé části se zabývám hardwarovým konceptem celého řídicího systému a nutných zařízení pro zónovou regulaci vytápění. V poslední třetí části se věnuji softwarovému konceptu respektive komunikační části mezi centrální jednotkou a akčními členy pro řízení jednotlivých topných okruhů.

\section{Cíl práce}
\begin{itemize}
\item Prostudovat problematiku podlahového vytápění při využití zónové regulace a její principy.
\item Navrhnout koncept centrální jednotky a dalších nutných zařízení pro zónovou regulace vytápění.
\item Navrhnout koncept komunikace centrální jednotky, lokálních termostatů a akčních členů pro řízení jednotlivých topných okruhů.
\end{itemize}