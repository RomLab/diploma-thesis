\chapter{Závěr}
Cílem práce bylo prostudovat problematiku zónového podlahového vytápění a navrhnout vlastní řešení pro řízení dílčích částí systému. Podle zadání diplomové práce se povedly splnit všechny body. Byla navržena jednotlivá zařízení pro dané části systému, včetně jejich mechanického umístění a ochranných krabiček. Některé části byly zakoupeny hotové a~případně došlo k jejich upravě. V rámci problematiky POE byl navržen DC/DC měnič pro PSE zařízení. Mezi jednotlivými částmi je zvolená vhodná komunikace a zejména jednoduchá rozšiřitelnost. Na centrální jednotce funguje open-source řídicí systém pro domácí automatizaci. Tento systém je neustále rozšiřován a aktualizován vývojářskou komunitou. Má již mnoho integrovaných částí pro řízení vytápění. Zároveň existuje již hotová mobilní aplikace, která je se systémem plně funkční, včetně upozorňování uživatelů pro zatopení v krbu. Systém se dá neustále rozšiřovat pro případné požadované úpravy, též software v~nástěnných snímačích prostorové teploty je rychle modifikovatelný. Vše tak bylo upravováno podle požadavků uživatelů domácnosti. Zároveň nebylo nutné ztrácet čas nad již hotovým základem softwaru centrální jednotky a více se věnovat samotné automatizaci. Využívají se nasbíraná data z jednotlivých místností v~rámci jejich vytápění a na základě nich se sestavuje velmi jednoduchá lineární predikci s předpovědí počasí pro úpravu časových teplotních úseků, aby v~požadovaný čas bylo dosaženo požadované teploty. 

Celý systém řízení se postupně vyvíjel. Původně bylo řízení vytápění podle chodbových termostatů a nepočítalo se se zónovým řízením podlahového vytápění. Následně se řízení rozšířilo o~zónové vytápění. Majitel domu chtěl primárně veškeré řešení drátové. Proto se nakonec vymýšlelo, jak rozvést kabely do jednotlivých místností. Vznikly primárně POE nástěnné snímače prostorové teploty, využilo se půdy pro rozvedení UTP kabelu do všech místností, kde to bylo možné a~zároveň nebyla ničena estetika místnosti. Pro místnosti, kde to nebylo možné, byly navrženy bezdrátové moduly s napájením ze síťových adaptérů, aby nebylo nutné se starat o výměnu baterií, požadavek majitele. Pro komunikaci bylo zvoleno MQTT, kde lze snadno výměnu dat měnit/upravovat/rozšiřovat. Pro kabelové řešení bylo zvoleno napájení po POE, pro bezdrátovou verzi se využila již stávající WiFi síť. Zařízení lze tak velmi rychle softwarově modifikovat a~přidávat, to se v~praxi velmi často stávalo. Na verzi s POE jsou poněkud vyšší náklady na součástky, nicméně ty nejdražší součástky mi zaslali výrobci jako vzorek. Jistou nevýhodou mohou být bezdrátové moduly z~pohledu bezdrátové komunikace a případných výpadků. Vzhledem k dobrému pokrytí WiFi sítě v~domě je komunikace bezproblémová.

DPS pro vstupy/výstup u centrální jednotky, I$^{2}$C rozdělovač, DPS pro signalizaci u krbů a~DPS v rozdělovačích pro podlahové vytápění byly ručně vyráběny pomocí fotocesty a mokrého leptání. DPS pro nástěnné snímače prostorové teploty byly vyrobeny ve specializované firmě. Následně ručně vyrobené i průmyslově vyrobené DPS byla vlastnoručně osazena a pájela. Pro plastové krabičky byl navrhnut 3D model a následně vytisknuté na 3D tiskárně. Zapojení rozvaděče v~technické místnosti i jiných částech byla provedena vlastnoručně.

V průběhu práce se narazilo na několik problémů. První z~nich byl správné zvolení impedančního zakončení diferenciální I$^{2}$C sběrnice, které po několikátém předělání nyní funguje bez problémů. Při tisknutí 3D krabičky vznikl problém se správným vytvořením podpěr v oblasti distančních sloupků, aby krabička šla vytisknout. Optimální řešení by bylo krabičku vytisknout bez distančních sloupků, respektive zvlášť a následně je nalepit.

