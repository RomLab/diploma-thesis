\chapter{Závěr}
Podle zadání diplomové práce se mi povedly splnit všechny body. Cílem práce bylo prostudovat problematiku zónového podlahového vytápění a navrhnout vlastní řešení pro řízení dílčích částí systému. Navrhl jsem koncept centrální jednotky a dalších částí pro zónovou regulaci vytápění. Navrhl jsem koncept komunikace centrální jednotky, lokálních \acrshort{nspt} a~akčních členů pro řízení jednotlivých otopných okruhů. Na základě konceptu jsem zvolil centrální jednotku, navrhl jednotlivá zařízení pro dané části systému, včetně jejich mechanického umístění a ochranných krabiček. Některé části jsem zakoupil hotové a~případně je upravil. V~rámci problematiky POE jsem navrhl DC/DC měnič pro PSE zařízení. Dalším splněným bodem zadání je zvolená vhodná komunikace a zejména jednoduchá rozšiřitelnost. Na centrální jednotce funguje open-source řídicí systém pro domácí automatizaci. Tento systém je neustále rozšiřován a aktualizován vývojářskou komunitou. Má již mnoho integrovaných částí pro řízení vytápění. V~rámci komunikace mezi centrální jednotkou a~\acrshort{nspt} byla zvolena komunikace pomocí MQTT, která je snadno nastavitelná a snadno rozšiřitelná. Mezi centrální jednotkou a akčními členy se využívá standardní I$^2$C sběrnice s úpravou pro komunikaci na delší vzdálenosti. Teplotní senzory jsou připojené na 1-Wire sběrnici. Zhotovil jsem \acrshort{nspt} do jednotlivých místností ve verzi s POE a WiFi s napájením pomocí síťového adaptéru. Pro \acrshort{nspt} jsem navrhl a~zhotovil krabičku pomocí 3D tiskárny. Pro ovládání jednotlivých otopných okruhů jsem zhotovil vlastní řešení s využitím zakoupených termoelektrických pohonů. V rámci celého systému se využívá automatizovaná část (inteligentní část) spočívající využívání teplotních plánů s možností jejich modifikování na základě teplotní predikce, kterou jsem do systému implementoval. Využívají se nasbíraná data z~jednotlivých místností v~rámci vytápění a~na základě nich se sestavuje velmi jednoduchá lineární predikci s předpovědí počasí pro úpravu časových teplotních úseků, aby v~požadovaný čas bylo dosaženo požadované teploty. Dále se využívá softwarová detekce otevřeného okna pro pozastavení vytápění v případě otevření okna. Vše jsem následně otestoval a nasadil v rodinném domě. Systém se dá neustále rozšiřovat pro případné požadované úpravy. Vše tak bylo upravováno podle požadavků uživatelů domácnosti. 

Celý systém řízení se postupně vyvíjel. Původně bylo řízení vytápění podle chodbových termostatů a nepočítalo se se zónovým řízením podlahového vytápění na které se následně přešlo. Majitel domu chtěl primárně veškeré řešení drátové. Proto se nakonec vymýšlelo, jak rozvést kabely do jednotlivých místností. Vznikly primárně POE \acrshort{nspt}, využilo se půdy pro rozvedení UTP kabelu do všech místností, kde to bylo možné a~zároveň nebyla ničena estetika místnosti. Pro místnosti, kde to nebylo možné, jsem navrhl bezdrátové moduly (WiFi) s~napájením ze síťových adaptérů, aby nebylo nutné se starat o výměnu baterií, požadavek majitele. Jistou nevýhodou mohou být bezdrátové moduly z~pohledu bezdrátové komunikace a případných výpadků. Vzhledem k dobrému pokrytí WiFi sítě v~domě je komunikace bezproblémová. Celkové náklady na celý systém jsou po zaokrouhlení 35 700 Kč. Nezanedbatelnou část částky tvoří termoelektrické pohony (celkově 2 × 12 pohonů pro 1. a 2. druhé patro), která činí přibližně 10 000 Kč. Cenové srovnání s~komerčními systému nedává úplně smysl. V mé cenové kalkulaci nejsou zahrnuté náklady například na samotný vývoj, kancelářské místnosti, splnění legislativních povinností, certifikace a~mnohé jiné. Cenová kalkulace se týká pouze součástek. Rozpis jednotlivých součástek a celková kalkulace je v příloze \ref{app:obsah-cd}.  Na verzi s POE jsou poněkud vyšší náklady na součástky, nicméně ty nejdražší součástky mi zaslali výrobci jako vzorek.

DPS pro vstupy/výstup u centrální jednotky, I$^{2}$C rozdělovač, DPS pro signalizaci u krbů a~DPS v rozdělovačích pro podlahové vytápění jsem vlastnoručně vyrobil pomocí fotocesty a mokrého leptání. DPS pro \acrshort{nspt} jsem vyrobil ve specializované firmě. Následně ručně vyrobené i průmyslově vyrobené DPS jsem osadil a zapájel. Pro plastové krabičky jsem navrhl 3D model a následně vytisknuté na 3D tiskárně. Zapojení rozvaděče v~technické místnosti i jiných částech jsem též provedl vlastnoručně.

V průběhu práce jsem narazil na několik problémů. Prvním problémem bylo správné zvolení impedančního zakončení diferenciální I$^{2}$C sběrnice, které po několikátém předělání nyní funguje bez problémů. Při tisknutí 3D krabičky vznikl problém se správným vytvořením podpěr v oblasti distančních sloupků, aby krabička šla vytisknout. Optimální řešení by bylo krabičku vytisknout bez distančních sloupků, respektive zvlášť a následně je nalepit. V rámci softwaru do \acrshort{nspt} se objevilo několik problémů například správné nastavení obnovení komunikace s centrální jednotkou v případě výpadku (odpojení centrální jednotky nebo odpojení \acrshort{nspt}). Tyto problémy se mi povedly vyřešit.

Celý systém funguje od jara roku 2020, kdy se postupně celý systém vyvíjel až do současného stavu. Systém funguje v objektu bez jediného výpadku a vzhledem k dnešním cenám energií je celá problematika o to zajímavější a navržená implementace dobrým příkladem alternativních řešení řízení. Celá tato práce, dále kód pro automatizaci HA, kód pro ESP32 (nástěnné snímače prostorové teploty), 3D krabička, schémata zapojení a DPS jsou veřejné na mém GitHub účtu \mbox{\url{https://github.com/RomLab}} v aktuální verzi. Systém hodlám dále vyvíjet.

