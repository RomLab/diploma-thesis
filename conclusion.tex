\chapter{Závěr}
Cílem práce bylo prostudovat problematiku zónového podlahové vytápění a navrhnout vlastní řešení pro řízení dílčích částí systému. Podle zadání diplomové práce se mi povedly splnit všechny body. Navrhl jsem jednotlivé zařízení pro dané části systému, včetně jejich mechanického umístění a ochranných krabiček. Některé části jsem již zakoupil hotové a~případně je upravil. Seznámil jsem se s problematikou POE a navrhl DC/DC měnič pro PSE zařízení. Mezi jednotlivými části jsem zvolil vhodnou komunikaci a zejména jednoduchou rozšiřovatelnost. Na centrální jednotce funguje open-source řídicí systém pro domácí automatizaci, tento systém je neustále rozšiřován a aktualizován vývojářskou komunitou. Má již mnoho integrovaných částí pro řízení vytápění. Zároveň existuje již hotová mobilní aplikace, která je se systémem plně funkční, včetně upozorňování uživatelů pro o zatopení v krbu. Systém se dá neustále rozšiřovat pro případné požadované úpravy, též software v~nástěnných snímačích prostorové teploty je rychle modifikovatelný. Vše jsem tak mohl upravit podle požadavků uživatelů domu. Zároveň jsem nemusel ztrácet čas nad již hotovým základem softwaru centrální jednotky a více se věnovat samotné automatizaci. Využívám nasbíraná data z jednotlivých místností v~rámci jejich vytápění a na základě nich sestavuji velmi jednoduchou lineární predikci s předpovědí počasí pro úpravu časových teplotních úseků, aby v~požadovaný čas bylo dosaženo požadované teploty. 

Celý systém řízení se postupně vyvíjel. Původně bylo řízení vytápění podle chodbových termostatů a nepočítalo se se zónovým řízením podlahové vytápění. Následně se řízení rozšířilo o zónové vytápění. Majitel domu chtěl primárně veškeré řešení drátové. Proto se nakonec vymýšlelo jak rozvést kabely do jednotlivých místností. Vznikly primárně POE nástěnné snímače prostorové teploty, využilo se půdy pro rozvedení UTP kabelu do všech místností, kde to bylo možné a zároveň to neničilo estetiku místnosti. Pro místnosti, kde to nebylo možné, jsem navrhl bezdrátové moduly s napájením ze síťových adaptérů, aby nebylo nutné se starat o výměnu baterií, požadavek majitele. Vhodnou formou komunikace bylo zvoleno MQTT, lze snadno výměnu dat měnit/upravovat/rozšiřovat, proto pro kabelové řešení jsem zvolit rovnou napájení po POE, pro bezdrátovou verzi jsem využil již stávající WiFi síť. Zařízení lze tak velmi rychle softwarově modifikovat a~přidávat, to se v~praxi velmi často stávalo. Na verzi s POE jsou poněkud vyšší náklady na součástky, nicméně ty nejdražší součástky mi zaslali výrobci jako vzorek. Jistou nevýhodu mohou být bezdrátové moduly z pohledu bezdrátové komunikace a případných výpadků, vzhledem k dobrému pokrytí WiFi sítě v~domě je komunikace bezproblémová.

DPS pro vstupy/výstup u centrální jednotky, I$^{2}$C rozdělovač, DPS pro signalizaci u krbů a DPS v rozdělovačích pro podlahové vytápění jsem vyráběl ručně pomocí fotocesty a mokrého leptání. DPS pro nástěnné snímače prostorové teploty jsem nechal vyrobit v specializované firmě. Následně ručně vyrobené i průmyslově vyrobené DPS jsem vlastnoručně osazoval a pájel. Pro plastové krabičky jsem navrhl 3D model a následně vytisk na 3D tiskárně. Zapojení rozvaděče v technické místnosti i jiných částech jsem též provedl já.

Průběžně jsem při řešení této práce narazil na několik problémů. První z~nich byl správné zvolení impedančního zakončení diferenciální I$^{2}$C sběrnice, které po několikátém předělání nyní funguje bez problémů. Při tisknutí 3D krabičky jsem narazil na problém se správným vytvořením podpěr v oblasti distančních sloupků, aby krabička šla vytisknout. Optimální řešení by bylo krabičku vytisknout bez distančních sloupků, respektive zvlášť a následně je nalepit.

