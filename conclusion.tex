\chapter{Závěr}
Celý systém řízení se vyvíjel. Původně bylo řízení vytápění podle chodbových termostatů. Následně se řízení rozšířilo o zónové vytápění. Majitel domu chtěl primárně veškeré řešení drátové. Proto vznikly primárně POE nástěnné snímače prostorové teploty, využilo se půdy pro rozvedení UTP kabelu do všech místností, kde to bylo možné a zároveň to neničilo estetiku místnosti. Pro místnosti, kde to nebylo možné jsem navrhl bezdrátové moduly s napájením ze síťových adaptérů, aby nebylo nutné se starat o výměnu baterií, požadavek majitele. Vhodnou formou komunikace bylo zvoleno MQTT, lze snadno výměnu dat měnit/upravovat/rozšiřovat, proto pro kabelové řešení jsem zvolit rovnou napájení po POE, pro bezdrátovou verzi jsem využil již stávající WiFi síť. Zařízení lze tak velmi rychle softwarově modifikovat a přidávat, to se v~praxi velmi často stávalo. Na verzi s POE jsou poněkud vyšší náklady na součástky, nicméně ty nejdražší součástky mi zaslala výrobci jako vzorek. Jistou nevýhodu mohou být bezdrátové moduly z pohledu bezdrátové komunikace a případných výpadků, vzhledem k dobrému pokrytí WiFi sítě v domě se komunikace jeví v pořádku.

%Jak již bylo řečeno v úvodu, současná verze dokumentu je teoretickou částí pro diplomovou práci. Podle zadání pro semestrální projekt se mi povedly splnit všechny body. Seznámil jsem se problematikou podlahového vytápění při využití zónové regulace. Navrhl jsem hardwarový koncept centrální jednotky a~dalších nutných zařízení pro zónovou regulaci vytápění. V  neposlední řadě jsem nastínil problematiku komunikace centrální jednotky, nástěnných snímačů prostorové teploty a akčních členů pro řízení jednotlivých otopných okruhů.

%V současné době realizuji hardwarové části a připravuji software jak pro řízení těchto zařízení, tak i pro zónovou regulaci podlahového vytápění. Navržený způsob komunikace nástěnných snímačů prostorové teploty s centrální jednotkou testuji a~připravuji jednotnou desku plošného spoje. Desku plošného spoje zónového regulátoru mám již navrženou, včetně výběru termoelektrických pohonů pro jednotlivé otopné okruhy.

%V další navazující praktické části (diplomová práce) popíši vybrané a~navržené komponenty/zařízení pro jednotlivé části z míněné v teoretické části. Dále uvedu softwarovou část pro zónovou regulaci na základě zvoleného řídicího systému Home Assistant. Navržený systém následně otestuji.