\chapter{Závěr}
Jak již bylo řečeno v úvodu, současná verze dokumentu je teoretickou částí pro diplomovou práci. Podle zadání pro semestrální projekt se mi povedly splnit všechny body. Seznámil jsem se problematikou podlahového vytápění při využití zónové regulace. Navrhl jsem hardwarový koncept centrální jednotky a~dalších nutných zařízení pro zónovou regulaci vytápění. V  neposlední řadě jsem nastínil problematiku komunikace centrální jednotky, nástěnných snímačů prostorové teploty a akčních členů pro řízení jednotlivých otopných okruhů. Navrhl jsem a realizoval zařízení pro ovládání prvků pro podlahové vytápění jako jsou čerpadla a signalizace teplot ze zásobníku otopné vody pro uživatele. V současné verzi je vytápěno podle centrálních nástěnných termostatů umístěných na chodbách v každém patře. Což samozřejmě není ideální řešení z pohledu vytápění a úspor, ale jedná se o základní verzi tohoto systému. Co se týče prototypů lokálních nástěnných snímačů pro zónovou regulaci, ty jsou stále ve vývoji.

Uvedl jsem aktuální verzi softwaru pro regulaci podlahové vytápění s jednotlivými módy. Dále jsem uvedl náznak rozšíření o zónovou regulaci, která je v současnosti stále ve vývoji.
